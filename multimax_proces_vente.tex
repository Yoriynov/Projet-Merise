\documentclass[a4paper,12pt]{article}
\usepackage[french]{babel}
\usepackage[T1]{fontenc}
\usepackage[utf8]{inputenc}
\usepackage{lmodern}
\usepackage{microtype}
\usepackage{hyperref}
\author{Easylia}
\begin{document}
\subsection*{Processus de vente Multimax}
La société Multimax vend sur Internet des articles informatiques. Elle a depuis peu également ouvert quatre magasins et prévoit d'en ouvrir davantage. 
Sur son site Web, la société vend de nombreux articles classés en catégories et sous-catégories.
\subsubsection*{Articles}
Chaque article dispose d'un titre, d'une description, d'un prix unitaire hors taxe (les gros clients disposent cependant d'un taux de réduction permanent qui a été négocié pour chaque article), de plusieurs photos, d'informations sur ses dimensions et sa masse, et d'indications sur le nombre d'articles en stock et leur éventuel délai de réapprovisionnement.

Plusieurs taux de TVA sont utilisés en fonction de l'article concerné.

L'historique de tous les articles ayant été proposés à la vente est conservé.

Un article peut éventuellement être constitué d'autres articles, lorsqu'un lot est proposé à la vente par exemple.
\subsubsection*{Note}
Les clients ayant déjà acheté un article peuvent laisser une note et un commentaire ;
ces notes et commentaires sont visibles par les acheteurs potentiels, sous réserve d'avoir été validés par un modérateur (ces modérateurs sont des salariés du département commercial disposant du rôle de modérateur)
(par ailleurs, on souhaite pouvoir garder un historique de l'évolution professionnelle des salariés au sein de l'entreprise, c'est à dire de leurs changements de département).
\subsubsection*{Client}
Lorsqu'un client effectue une commande, il doit s'identifier. Si le client n'a jamais acheté sur le site, il est invité à saisir ses coordonnées complètes de facturation.
\subsubsection*{Adresse de livraison}
Multimax permet également d'historiser de multiples adresses de livraison pour un client (que les clients peuvent éventuellement supprimer depuis leur espace client).
\subsubsection*{Commande}
Chaque commande peut porter sur un ou plusieurs produits, différents ou pas.
Un historique de toutes les commandes doit être conservé.
\subsubsection*{Modes de livraison}
Plusieurs modes de livraison sont proposés pour une commande ;
le coût de la livraison étant calculé automatiquement en fonction d'un coefficient variable pour chaque mode de livraison et de la masse totale des produits de la commande.
Le client peut également aller chercher sa commande dans une des boutiques de l'entreprise ;
dans ce cas, les frais de livraison sont offerts.
\subsubsection*{Mode de paiement}
Une fois le compte-rendu de la commande accepté par le client, celui-ci est invité à choisir son mode de paiement. Deux modes sont possibles : par chèque - dans ce cas, la commande n'est validée par Multimax qu'après réception, enregistrement (date et numéro) et encaissement du chèque - ou par carte de paiement - dans ce cas, le client est redirigé vers le site Web du prestataire de paiement qui renvoie un code selon que le paiement est accepté ou pas.

Une fois la commande validée, le client reçoit un courriel résumant sa commande en provenance du service achats.
\subsubsection*{Préparation commande}
Le service achats transmet alors aux entrepôts une demande de préparation de commande. Si les produits sont en stocks, la commande est préparée immédiatement. Si les produits ne sont pas en stocks, un courriel est envoyé au client pour lui indiquer la date prévisionnelle d'expédition de sa commande.
\subsubsection*{Expédition}
Une fois la commande préparée, les articles sont expédiés selon le mode d'expédition choisi par le client lors de la commande.
\subsubsection*{Retrait}
Si le client avait choisi un retrait en magasin, les commandes prêtes pour expédition sont stockées durant un jour, jusqu'à remplir complètement un camion à destination du magasin de retrait.
Dans tous les cas, lorsque la commande quitte les entrepôts, le client reçoit un courriel de confirmation d'expédition, comportant la date de livraison prévue, variable selon le mode d'expédition qui avait été choisi.
\subsubsection*{Facturation}
Le service facturation est également informé de l'expédition de la commande. Celui-ci prépare alors la facture correspondante qui est envoyée au client par courriel. Les clients peuvent éventuellement choisir une facturation mensuelle où chaque facture peut, dès lors, porter sur plusieurs commandes faites dans le mois.
\subsubsection*{Gestion de stock}
Par ailleurs, lorsque le stock d'un article passe en dessous d'un seuil critique différent pour chaque article, Multimax déclenche automatiquement une commande de cet article auprès de ses fournisseurs.
Pour cela, le service approvisionnement de Multimax émet d'abord une demande de cotation auprès de ses fournisseurs réguliers ; ceux-ci ont 24 heures pour y répondre.
Multimax choisi ensuite le meilleur prix, passe commande auprès de ce fournisseur, et informe les autres qu'ils n'ont pas été choisis.
\subsubsection*{Réception}
À réception des produits aux entrepôts de Multimax, une vérification de la commande initiale est effectuée.
Si la commande n'est pas conforme, elle est réexpédiée au fournisseur, et une note négative est indiquée dans la fiche de ce fournisseur, afin qu'il ne soit plus priorisé dans les prochains choix de fournisseurs.
Si la commande est conforme, celle-ci est enregistrée et ajoutée aux stocks. Les stocks des produits concernés sont mis à jour.
\subsubsection*{Paiement fournisseurs}
Les entrepôts de Multimax émettent dans ce cas une demande de paiement auprès du service comptable. Celui-ci procède au règlement, après réception de la facture correspondante par le fournisseur, si celle-ci est conforme à la demande de paiement émise par les entrepôts.
\end{document}