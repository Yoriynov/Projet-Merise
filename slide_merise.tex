\documentclass[bigger]{beamer}
\usepackage[utf8]{inputenc}
\usepackage[T1]{fontenc}

\usepackage{graphicx}


\usebackgroundtemplate%
{%
    \includegraphics[width=\paperwidth,height=\paperheight]{background.jpg}%
}



\begin{document}

{
\usebackgroundtemplate{\includegraphics[width=\paperwidth]{presentation.jpg}}%
\begin{frame}{}
\end{frame}
}


\begin{frame}{Merise}

Qu'est-ce que la méthode Merise ?

\begin{itemize}
\item<2-4> Le niveau conceptuel
\item<3,4> Le niveau logique (ou organisationnel)
\item<4> Le niveau physique
\end{itemize}
\end{frame}



\begin{frame}{Merise}

\begin{itemize}
\item<1> Quelle est son origine ?
\item<2> Où cette méthode est-elle utilisée ?
\item<3> Pourquoi peut-ont être amené à utiliser la méthode Merise ?
\item<4> Comment ça marche ?

\end{itemize}
\end{frame}

\begin{frame}{Merise}
FIN ! 

Les questions sont bienvenues !
\end{frame}

\end{document}